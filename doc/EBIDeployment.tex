\documentclass[a4paper,10pt]{article}
\usepackage[utf8]{inputenc}

%opening
\title{EBI-MuG PMES}
\author{Workflows and Distributed computing}

\begin{document}

\maketitle

\section{Acces}
OpenStack acces:
\begin{verbatim}
URL: https://extcloud05.ebi.ac.uk/dashboard/project/
username: XXX@bsc.es
password: XXXX
\end{verbatim}

First acces to the VM with floating point.
\begin{verbatim}
 ssh -vv -i XXX.pem -l ubuntu 193.62.52.104
\end{verbatim}
Then acces to the VM which is hosting the PMES service:
\begin{verbatim}
 ssh -i XXX.pem -l ubuntu 192.168.0.27
\end{verbatim}
Finally change to the user PMES.
\begin{verbatim}
 sudo su pmes
\end{verbatim}

\section{OCCI}
The occi endpoint is \texttt{https://extcloud05.ebi.ac.uk:8787/occi1.1/} \\
The occi authentication mode is token. To get the token you have to do the following steps:
\begin{itemize}
 \item source a file with the following parameters:
\begin{verbatim}
export OS_NO_CACHE=True
export OS_CLOUDNAME=overcloud
export OS_AUTH_URL=https://extcloud05.ebi.ac.uk:13000/v2.0
export NOVA_VERSION=1.1
export COMPUTE_API_VERSION=1.1
export OS_USERNAME=XXX@bsc.es
export OS_PASSWORD=XXXX
export OS_TENANT_NAME=BSC-MuG
\end{verbatim}
\item execute the following command:
\begin{verbatim}
 curl -v  -s -X POST $OS_AUTH_URL/tokens -H "Content-Type: application/json" 
 -d '{"auth": {"tenantName": "'"$OS_TENANT_NAME"'", "passwordCredentials": 
 {"username": "'"$OS_USERNAME"'", "password": 
"'"$OS_PASSWORD"'"}}}' | python3 -m json.tool >> data.json
\end{verbatim}
\item get token from data.json and export the following variable:
\begin{verbatim}
 export OS_TOKEN=toke_from_data.json
\end{verbatim}

With the token you can execute occi commands with auth mode token. For example:
\begin{verbatim}
 occi --endpoint https://extcloud05.ebi.ac.uk:8787/occi1.1/ --auth token 
 -q $OS_TOKEN --action list --resource resource_tpl
\end{verbatim}

\end{itemize}

\section{PMES}
PMES service is already installed using tomcat7. The endpoint is \texttt{http://localhost:8080/pmes/pmes/}.

You can call the service using curl or you can call the service using the python script \texttt{/home/pmes/pmes/scripts/curlPmesApi.py}.

\begin{verbatim}
 # api call getSystemStatus
 python3 curlPmesApi.py getSystemStatus
 # api call getActivityReport
 python3 curlPmesApi.py getActivityReport job_id
 # api call terminateActivity
 python3 curlPmesApi.py terminateActivity job_id
 # api call createActivity
 python3 curlPmesApi.py createActivity createVM.json
\end{verbatim}

\texttt{createVM.json} is a json file with a job definition. For example:
\begin{verbatim}
 [{ "jobName": "HelloTest2_584817558cb7550b5e9970b0",
           "wallTime": "5",
           "minimumVMs": "1",
           "maximumVMs": "1",
           "limitVMs": "1",
           "initialVMs": "1",
           "memory": "1.0",
           "cores": "1",
           "disk" : "1.0",
           "inputPaths": ["/home/"],
           "outputPaths": ["/home/"],
           "mountPath":"",
           "numNodes": "1",
           "user":
              { "username": "lcodo",
                "credentials":
                { "pem": "",
                  "key": "",
                  "token": "6718197acbb04f7d9695f832cfb2e38f" }
              },
           "img": 
              { "imageName": "os_tpl#4f916ede-218b-47e4-93aa-b795a5acf813", 
                "imageType": "resource_tpl#721112dd-2f33-40eb-8975-7bd34dbabfc8"
              },
           "app":
              { "name": "HelloTest2",
                "target": "/home/pmes/testSimple",
                "source": "launch.sh",
                "args": { "val1": "Hola", "val2": "Mundo" } ,
                "type": "COMPSs"
              },
           "compss_flags": {}
        }]
\end{verbatim}



\end{document}
